%******************************FORMATIERUNG****************************************
\documentclass[a4paper, 12pt]{article}
\usepackage{scrpage2}
\usepackage{todonotes}
\usepackage{amssymb}
\usepackage{amsmath}
\usepackage{caption}
\pagestyle{scrheadings}
\clearscrheadfoot
\usepackage[ngerman]{babel}
\usepackage[utf8]{inputenc}
\usepackage{graphicx}
\usepackage[space]{grffile}
\usepackage{setspace}
\usepackage[T1]{fontenc}
\newcommand{\changefont}[3]{
\fontfamily{#1} \fontseries{#2} \fontshape{#3} \selectfont}
\usepackage{datetime}
\newdateformat{monthyeardate}{%
  \monthname[\THEMONTH] \THEYEAR}
\usepackage{geometry}
\geometry{verbose,a4paper,tmargin=30mm,bmargin=30mm,lmargin=35mm,rmargin=25mm}
\usepackage[numbers,square]{natbib}
\usepackage[
	breaklinks=true,
	pdfauthor={Laura Anger, Vera Brockmeyer, Paul Berning, Lukas Kolhagen},
	pdftitle={Masterprojekt - MArC},
	pdftoolbar=true,
	pdfsubject={Mixed Reality Architecture Composer},
	colorlinks=true,
	linkcolor=blue,
	citecolor=blue,
	urlcolor=blue,
	linktocpage=true
	]{hyperref}
\usepackage{algorithmicx}
\usepackage{multicol}
\usepackage{multirow}
\usepackage[ruled]{algorithm}
\usepackage{algpseudocode}
\usepackage{pdfpages}
\usepackage[
	font=small,
	labelfont=bf, 
	format=plain,
	indention=1cm
	]{caption}
\setlength{\parindent}{0pt} 
\setlength{\parskip}{.5em}
\usepackage{color}
\definecolor{myColor}{rgb}{0.8,0.8,0.8}
\newcommand{\Absatzbox}[1]{\parbox[0pt][2em][c]{0cm}{}}
\usepackage{listings}

%*****************************ENDE FORMATIERUNG****************************************

\begin{document}
%\linespread{1.2}
%\changefont{ppl}{m}{n}

\thispagestyle{empty}
\begin{center}
	\includegraphics[width=5cm]{Bilder/logo_TH}\\[12ex]
	{\Huge\textbf{Cost Plan}}\\[8ex]
	\rule{.8\textwidth}{.2pt}
	{\Large\\[1ex] \textbf{Interaction Lab}}\\
	\rule{.8\textwidth}{.2pt}\\[10ex]
	written by\\[2ex]
	\begin{tabular}{ll}
		Laura Anger &(Matrikelnr. 11086356)\\ 
		Vera Brockmeyer &(Matrikelnr. 11077082)\\
		Anna Bolder &(Matrikelnr. 11083451)\\
		Britta Boerner &(Matrikelnr. 11070843)\\
	\end{tabular}\\[10ex]
	\textbf{Interactive Systems in SS 2017}\\			
\end{center}
\vfill
\begin{flushleft}
	{\bf Supervisor:}\\
	Prof. Dr. Stefan Michael Grünvogel\\
	Institute for Media- and Phototechnology
\end{flushleft}
\newpage
\newpage
\begin{tabular}{| p{\textwidth/4}| p{\textwidth/4}| p{\textwidth/4}| p{\textwidth/4} |}
	\hline 
	\textbf{Quality goal} &	\textbf{Criterion} & \textbf{Method} & \textbf{Controlling} \\ \hline \hline
%%overall -VR
small latency & 20 ms maximum & simple rooms and the calculation should not be too expensive & testing, fps rate shown in unity \\ \hline
no dropouts & no blackframe or errors in the unity project & no expensive calculations, not do much calculations parallel & testing \\ \hline
immersive & scene should be as real as possible & realistic objects, moving like in reality & testing \\ \hline  
%% Scene
learning & ability to learn and test all interactions & learning room, simple, without tasks, always start in this room & testing \\ \hline
realistic & scene should be as real as possible & realistic objects, moving like in reality, textures & testing \\ \hline  
different sizes of objects & small as well as big objects within the scenes & create a room where it is natural that there are different sizes of objects (for example supermarket) & testing and looking for all sizes \\ \hline  
%% selection near/far?
accuracy of selection & user grab the right object, the space that selects an object is not that big? & user is using best interaction for object, interactions are as good as possible implemented & usability study, tasks, scene(?) recognizes if the correct object is picked \\ \hline  
correct movement of an object & if the object is grabbed, the movement of the object is according to the hand & parenting the object to the movement of the controller, if it is grabbed & testing, measurement of the positions? \\ \hline 
duration to finish a specific task & time, tasks & measure time between the time when tasks starts and time when the tasks is finished & time, time should not be to long? \\ \hline  
successful fulfilment of a specific task & tasks, correct objects & scene(?) recognizes if the correct object is picked, person from the outside looks for the correction??? & implemented correction, study, testing \\ \hline 
%%  tasks
understandability of tasks & object, term, assignment of tasks & clear, easy to understand, use common objects & testing \\ \hline 

%% hardware
correct use of the HTC Vive??? 
%% Dates
milestones? 

\end{tabular}

%%  Quality goal   &	   Criterion   &    Method    &    Controlling
\end{document}