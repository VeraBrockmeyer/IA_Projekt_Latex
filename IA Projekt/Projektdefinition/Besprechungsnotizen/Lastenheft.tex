%******************************FORMATIERUNG****************************************
\documentclass[a4paper, 12pt]{article}
\usepackage{scrpage2}
\usepackage{todonotes}
\usepackage{amssymb}
\usepackage{amsmath}
\usepackage{caption}
\pagestyle{scrheadings}
\clearscrheadfoot
\setheadsepline{.5pt}
\setfootsepline{.5pt}
\automark[section]{chapter}
\ihead{\headmark}
\ohead{\pagemark}
\usepackage[ngerman]{babel}
\usepackage[utf8]{inputenc}
\usepackage{graphicx}
\usepackage[space]{grffile}
\usepackage{setspace}
\usepackage[T1]{fontenc}
\newcommand{\changefont}[3]{
\fontfamily{#1} \fontseries{#2} \fontshape{#3} \selectfont}
\usepackage{datetime}
\newdateformat{monthyeardate}{%
  \monthname[\THEMONTH] \THEYEAR}
\usepackage{geometry}
\geometry{verbose,a4paper,tmargin=30mm,bmargin=30mm,lmargin=35mm,rmargin=25mm}
\usepackage[numbers,square]{natbib}
\usepackage[
	breaklinks=true,
	pdfauthor={Laura Anger, Vera Brockmeyer, Paul Berning, Lukas Kolhagen},
	pdftitle={Masterprojekt - MArC},
	pdftoolbar=true,
	pdfsubject={Mixed Reality Architecture Composer},
	colorlinks=true,
	linkcolor=blue,
	citecolor=blue,
	urlcolor=blue,
	linktocpage=true
	]{hyperref}
\usepackage{algorithmicx}
\usepackage{multicol}
\usepackage{multirow}
\usepackage[ruled]{algorithm}
\usepackage{algpseudocode}
\usepackage{pdfpages}
\usepackage[
	font=small,
	labelfont=bf, 
	format=plain,
	indention=1cm
	]{caption}
\setlength{\parindent}{0pt} 
\setlength{\parskip}{.5em}
\usepackage{color}
\definecolor{myColor}{rgb}{0.8,0.8,0.8}
\newcommand{\Absatzbox}[1]{\parbox[0pt][2em][c]{0cm}{}}
\usepackage{listings}

%*****************************ENDE FORMATIERUNG****************************************

\begin{document}
%\linespread{1.2}
%\changefont{ppl}{m}{n}

%INHALTSVERZEICHNIS
%\pagenumbering{arabic}
%\input{main/Inhaltsverzeichnis.tex}

\thispagestyle{empty}
\begin{center}
			\includegraphics[width=5cm]{Bilder/logo_TH}\\[12ex]
			{\Huge\textbf{Projektdefinition}}\\[8ex]
			\rule{.8\textwidth}{.2pt}
			{\Large\\[1ex] \textbf{VR-Interface-LAB for Grabbing Interaction}}\\
			%{\textbf{M}ixed Reality \textbf{Ar}chitecture \textbf{C}omposer}\\
			\rule{.8\textwidth}{.2pt}\\[10ex]
			von\\[2ex]
			\begin{tabular}{ll}
			Laura Anger &(Matrikelnr. 11086356)\\ 
			Vera Brockmeyer &(Matrikelnr. 11077082)\\
			Anna Bolder &(Matrikelnr. 11083451)\\
			Britta Boerner &(Matrikelnr. 11070843)\\
			\end{tabular}\\[10ex]
			\textbf{Interactive Systems}\\
			im \textbf{SS 17}\\			
			\end{center}
			\vfill
			\begin{flushleft}
			{\bf Betreuer:}\\
			Prof. Dr. Stefan Michael Grünvogel\\
			Institut für Medien- und Phototechnik
			\end{flushleft}
	\newpage
	\tableofcontents
	\newpage

\section{Problem Analyse}
\subsection{Problembeschreibung}
Welches Problem tritt konkret auf? \\

- es wird zu Forschungszwecken ein VR Labor benötigt mit welchem man interaktions technicken ausprobieren und vergleichen kann
-für Lehrzwecke: Studien, Demostration 
- Interaktionstechniken von Gerät und System abhängig
\\

Wie macht sich das Problem bemerkbar? 
- keine Möglichkeit Studenten die Interaktionen im Vergleich zuzeigen.
-
\\
In welchen Unternehmensbereichen bzw. bei welchen Produkten oder Prozessen tritt das Problem auf? 

- Unibetrieb oder Forschung und Entwicklung
\\
Auf welche Weise können die derzeitige Situation bzw. der betroffene Prozess im Detail erhoben und dargestellt werden (IST-Analyse)?

-\todo{Britta Recherche}
\\
Seit wann tritt das Problem auf?
- durch veröffentlichung der Consumer geräte wie Oculus und Vive rein werden immer mehr Techniken und Methoden auf den Markt gebracht. Die Nachfrage steigt stetig
-1966 erstet HMD und Datenhandschuh (Sutherland and Goertz)
\\
Welche betriebswirtschaftlichen Auswirkungen hat das Problem? 

- durch in Entwicklung und Evaluation der Techniken kann in Zukunft Produkte entwickelt werden die Benutzerfreundlicher sind und somit mehr gefragt
\\
Welche Personen sind beteiligt? 
- Lehrstuhl  mit Studenten, Professoren und ggf. Entwickler
\\
Welche Sachmittel kommen gegenwärtig zum Einsatz? 
-Oculus System
- Vive
- Controller
- Handschuh
- Motion Capturing Hände
\\
Wie laufen die Prozesse derzeit ab? 
- systemabhängig und anwendungsabhängig
\\
In welchem wirtschaftliche und technischen Umfeld wird das Problem beobachtet?
- Lehrbetrieb
\subsection{Ursachenanalyse}
Wie konnte es zu der Abweichung zwischen "Ist" und "Soll" kommen? 
- Systeme haben unterschiedliche Interaktionsmöglichkeiten
- keine Standards und Richtilinen
-Unternehmen entwickeln unabhänige Systeme
\\
Hängen die Ursachen mit den beteiligten Personen zusammen? 
-Unternehmen entwickeln unabhänige Systeme
\\
Liegt die Ursachen für das Problem –in der Organisation des Unternehmens oder des Geschäftsprozesses? – in den verwendeten Sachmitteln, Verfahren oder techn. Hilfsmitteln?
\\
Sind Veränderungen im Umfeld für die Entstehung des Problems verantwortlich – und wenn ja, welche? 
- Boom Markt von Consumer Produkten
\section{Projektziele}
- zwei Testszenen
- ein Lern und Testraum wo alle angebotenen Interaktionen ausprobiert werden
- ein Supermarkt mit Einkaufsliste oder ähnliches in dem diverse Aufgaben mit unterschiedlichen Schwierigkeitsgraden, Greifentfernungen und Objekten
- optionale Erweiterung um weitere Interaktionstypen (Laden anderer Szenen)
- Nur für Vive
- nur Interaktionen mit Controller oder mit Vive HMD
- sechs Interaktionstypen werden implementiert

- Release 15.07.2017

\subsection{Anforderungskatalog}
- Interaktionen zur Auswahl im Nahbereich
- Interaktionen zur Auswahl im Fernbereich
- alle Interaktionen sollen greifen und positionieren
- Parametrisierung ?????

\subsection{Lastenheft}
\section{Entwurf von Projektergebnissen ("Lösungskonzept") }
Design oder Skizze der GUI eine Animationssoftware
Zeichnung des Netzwerkes in einem Fernsehstudio 
\section{Durchführbarkeitsanalyse}
Machbarkeit: Projekt muss tatsächlich realisierbar sein 
\\
Projektrisiko: Das Risiko muss überschaubar sein 
\\
Wirtschaftlichkeit: Aufwand und Erfolg müssen in einem angemessen Verhältnis stehen 
\section{Projektvertrag}
\section{Projektorganisation}
\subsection{Projektleitung und Projektteam}
\subsection{Projektinfrastruktur}
Räumlichkeiten
\\
Größe, Lagen, Eignung für Vorhaben 
Arbeitsmittel
\\
Computer in ausreichender Zahl und mit notwendigen 
Leistungsmerkmalen vorhanden, weitere technische Hilfsmittel
Dienstleistungen
\\
Sekretariat, andere Unternehmenseinheiten

\subsection{Projektinformationssystem}
Projektordner 
\\
Wird zu Projektbeginn in Verantwortung des Projektleiters angelegt
\\
Dient der gesamten Projektdokumentation – schriftl. Erfassung des Projektprozesses und Produktentstehung E-Mail
\\
Standardisierte "Betreff" -Zeile zur Identifikation 
\\
Verwenden von Empfangsbestätigungen Intranet, Shared Webspace 
\\
Kollaboratives Arbeiten und Austausch von Dokumenten\\
Teambesprechungsroutinen 
\\
Stellen sicher, dass Mitarbeiter der verschiedenen Bereiche regelmäßig über Entwicklung in anderen Teilbereichen des Projekts informiert sind Reviews (Überprüfungen)
\\
Projektmitarbeiter informieren gesamtes Projektteam über ihre Zwischenergebnisse 
\\
Damit: Alle Projektmitarbeiter werden über alle Details informiert
\\
Anwesenheit ist in der Regel Pflicht und wird dokumentiert Regelsystem
\\
Transparent und einfach, an das sich alle halten 
\newpage
\bibliographystyle{plain}
\bibliography{main}
\end{document}