%******************************FORMATIERUNG****************************************
\documentclass[a4paper, 12pt]{article}
\usepackage{scrpage2}
\usepackage{todonotes}
\usepackage{amssymb}
\usepackage{amsmath}
\usepackage{caption}
\pagestyle{scrheadings}
\clearscrheadfoot
\setheadsepline{.5pt}
\setfootsepline{.5pt}
\automark[section]{chapter}
\ihead{\headmark}
\ohead{\pagemark}
\usepackage[ngerman]{babel}
\usepackage[utf8]{inputenc}
\usepackage{graphicx}
\usepackage[space]{grffile}
\usepackage{setspace}
\usepackage[T1]{fontenc}
\newcommand{\changefont}[3]{
\fontfamily{#1} \fontseries{#2} \fontshape{#3} \selectfont}
\usepackage{datetime}
\newdateformat{monthyeardate}{%
  \monthname[\THEMONTH] \THEYEAR}
\usepackage{geometry}
\geometry{verbose,a4paper,tmargin=30mm,bmargin=30mm,lmargin=35mm,rmargin=25mm}
\usepackage[numbers,square]{natbib}
\usepackage[
	breaklinks=true,
	pdfauthor={Laura Anger, Vera Brockmeyer, Paul Berning, Lukas Kolhagen},
	pdftitle={Masterprojekt - MArC},
	pdftoolbar=true,
	pdfsubject={Mixed Reality Architecture Composer},
	colorlinks=true,
	linkcolor=blue,
	citecolor=blue,
	urlcolor=blue,
	linktocpage=true
	]{hyperref}
\usepackage{algorithmicx}
\usepackage{multicol}
\usepackage{multirow}
\usepackage[ruled]{algorithm}
\usepackage{algpseudocode}
\usepackage{pdfpages}
\usepackage[
	font=small,
	labelfont=bf, 
	format=plain,
	indention=1cm
	]{caption}
\setlength{\parindent}{0pt} 
\setlength{\parskip}{.5em}
\usepackage{color}
\definecolor{myColor}{rgb}{0.8,0.8,0.8}
\newcommand{\Absatzbox}[1]{\parbox[0pt][2em][c]{0cm}{}}
\usepackage{listings}

%*****************************ENDE FORMATIERUNG****************************************

\begin{document}
%\linespread{1.2}
%\changefont{ppl}{m}{n}

%INHALTSVERZEICHNIS
%\pagenumbering{arabic}
%\input{main/Inhaltsverzeichnis.tex}

\thispagestyle{empty}
\begin{center}
			\includegraphics[width=5cm]{Bilder/logo_TH}\\[12ex]
			{\Huge\textbf{Projektdefinition}}\\[8ex]
			\rule{.8\textwidth}{.2pt}
			{\Large\\[1ex] \textbf{VR-Interface-LAB for Grabbing Interaction}}\\
			%{\textbf{M}ixed Reality \textbf{Ar}chitecture \textbf{C}omposer}\\
			\rule{.8\textwidth}{.2pt}\\[10ex]
			von\\[2ex]
			\begin{tabular}{ll}
			Laura Anger &(Matrikelnr. 11086356)\\ 
			Vera Brockmeyer &(Matrikelnr. 11077082)\\
			Anna Bolder &(Matrikelnr. 11083451)\\
			Britta Boerner &(Matrikelnr. 11070843)\\
			\end{tabular}\\[10ex]
			\textbf{Interactive Systems}\\
			im \textbf{SS 17}\\			
			\end{center}
			\vfill
			\begin{flushleft}
			{\bf Betreuer:}\\
			Prof. Dr. Stefan Michael Grünvogel\\
			Institut für Medien- und Phototechnik
			\end{flushleft}
	\newpage
	\tableofcontents
	\newpage

\section{Problem Analyse}
\subsection{Problembeschreibung}
Welches Problem tritt konkret auf? \\
Wie macht sich das Problem bemerkbar? 
\\
In welchen Unternehmensbereichen bzw. bei welchen Produkten oder Prozessen tritt das Problem auf? 
\\
Auf welche Weise können die derzeitige Situation bzw. der betroffene Prozess im Detail erhoben und dargestellt werden (
IST-Analyse)? 
\\
Seit wann tritt das Problem auf? 
\\
Welche betriebswirtschaftlichen Auswirkungen hat das Problem? 
\\
Welche Personen sind beteiligt? 
\\
Welche Sachmittel kommen gegenwärtig zum Einsatz? 
\\
Wie laufen die Prozesse derzeit ab? 
\\
In welchem wirtschaftliche und technischen Umfeld wird das Problem beobachtet?
\subsection{Ursachenanalyse}
Wie konnte es zu der Abweichung zwischen "Ist" und "Soll" kommen? 
\\
Hängen die Ursachen mit den beteiligten Personen zusammen? 
\\
Liegt die Ursachen für das Problem –in der Organisation des Unternehmens oder des Geschäftsprozesses? – in den verwendeten Sachmitteln, Verfahren oder techn. Hilfsmitteln?
\\
Sind Veränderungen im Umfeld für die Entstehung des Problems verantwortlich – und wenn ja, welche? 
\section{Projektziele und Anforderungskatalog}
\subsection{Projektziele}
\subsection{Lastenheft}
\section{Entwurf von Projektergebnissen ("Lösungskonzept") }
Design oder Skizze der GUI eine Animationssoftware
Zeichnung des Netzwerkes in einem Fernsehstudio 
\section{Durchführbarkeitsanalyse}
Machbarkeit: Projekt muss tatsächlich realisierbar sein 
\\
Projektrisiko: Das Risiko muss überschaubar sein 
\\
Wirtschaftlichkeit: Aufwand und Erfolg müssen in einem angemessen Verhältnis stehen 
\section{Projektvertrag}
\section{Projektorganisation}
\subsection{Projektleitung und Projektteam}
\subsection{Projektinfrastruktur}
Räumlichkeiten
\\
Größe, Lagen, Eignung für Vorhaben 
Arbeitsmittel
\\
Computer in ausreichender Zahl und mit notwendigen 
Leistungsmerkmalen vorhanden, weitere technische Hilfsmittel
Dienstleistungen
\\
Sekretariat, andere Unternehmenseinheiten

\subsection{Projektinformationssystem}
Projektordner 
\\
Wird zu Projektbeginn in Verantwortung des Projektleiters angelegt
\\
Dient der gesamten Projektdokumentation – schriftl. Erfassung des Projektprozesses und Produktentstehung E-Mail
\\
Standardisierte "Betreff" -Zeile zur Identifikation 
\\
Verwenden von Empfangsbestätigungen Intranet, Shared Webspace 
\\
Kollaboratives Arbeiten und Austausch von Dokumenten\\
Teambesprechungsroutinen 
\\
Stellen sicher, dass Mitarbeiter der verschiedenen Bereiche regelmäßig über Entwicklung in anderen Teilbereichen des Projekts informiert sind Reviews (Überprüfungen)
\\
Projektmitarbeiter informieren gesamtes Projektteam über ihre Zwischenergebnisse 
\\
Damit: Alle Projektmitarbeiter werden über alle Details informiert
\\
Anwesenheit ist in der Regel Pflicht und wird dokumentiert Regelsystem
\\
Transparent und einfach, an das sich alle halten 
\newpage
\bibliographystyle{plain}
\bibliography{main}
\end{document}