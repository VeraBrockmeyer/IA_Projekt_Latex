%******************************FORMATIERUNG****************************************
\documentclass[a4paper, 12pt]{article}
\usepackage{scrpage2}
\usepackage{todonotes}
\usepackage{amssymb}
\usepackage{amsmath}
\usepackage{caption}
\pagestyle{scrheadings}
\clearscrheadfoot
\usepackage[ngerman]{babel}
\usepackage[utf8]{inputenc}
\usepackage{graphicx}
\usepackage[space]{grffile}
\usepackage{setspace}
\usepackage[T1]{fontenc}
\newcommand{\changefont}[3]{
\fontfamily{#1} \fontseries{#2} \fontshape{#3} \selectfont}
\usepackage{datetime}
\newdateformat{monthyeardate}{%
  \monthname[\THEMONTH] \THEYEAR}
\usepackage{geometry}
\geometry{verbose,a4paper,tmargin=30mm,bmargin=30mm,lmargin=35mm,rmargin=25mm}
\usepackage[numbers,square]{natbib}
\usepackage[
	breaklinks=true,
	pdfauthor={Laura Anger, Vera Brockmeyer, Paul Berning, Lukas Kolhagen},
	pdftitle={Masterprojekt - MArC},
	pdftoolbar=true,
	pdfsubject={Mixed Reality Architecture Composer},
	colorlinks=true,
	linkcolor=blue,
	citecolor=blue,
	urlcolor=blue,
	linktocpage=true
	]{hyperref}
\usepackage{algorithmicx}
\usepackage{multicol}
\usepackage{multirow}
\usepackage[ruled]{algorithm}
\usepackage{algpseudocode}
\usepackage{pdfpages}
\usepackage[
	font=small,
	labelfont=bf, 
	format=plain,
	indention=1cm
	]{caption}
\setlength{\parindent}{0pt} 
\setlength{\parskip}{.5em}
\usepackage{color}
\definecolor{myColor}{rgb}{0.8,0.8,0.8}
\newcommand{\Absatzbox}[1]{\parbox[0pt][2em][c]{0cm}{}}
\usepackage{listings}

%*****************************ENDE FORMATIERUNG****************************************

\begin{document}
%\linespread{1.2}
%\changefont{ppl}{m}{n}

\thispagestyle{empty}
\begin{center}
			\includegraphics[width=5cm]{Bilder/logo_TH}\\[12ex]
			{\Huge\textbf{Projektdefinition}}\\[8ex]
			\rule{.8\textwidth}{.2pt}
			{\Large\\[1ex] \textbf{VR-Interface-LAB for Grabbing Interaction}}\\
			%{\textbf{M}ixed Reality \textbf{Ar}chitecture \textbf{C}omposer}\\
			\rule{.8\textwidth}{.2pt}\\[10ex]
			von\\[2ex]
			\begin{tabular}{ll}
			Laura Anger &(Matrikelnr. 11086356)\\ 
			Vera Brockmeyer &(Matrikelnr. 11077082)\\
			Anna Bolder &(Matrikelnr. 11083451)\\
			Britta Boerner &(Matrikelnr. 11070843)\\
			\end{tabular}\\[10ex]
			\textbf{Interactive Systems}\\
			im \textbf{SS 17}\\			
			\end{center}
			\vfill
			\begin{flushleft}
			{\bf Betreuer:}\\
			Prof. Dr. Stefan Michael Grünvogel\\
			Institut für Medien- und Phototechnik
			\end{flushleft}
	\newpage
\newpage


\textbf{\Large Project Order of Interaction Lab}\\


\textbf{Project Manager:} Vera Brockmeyer\\


\textbf{Problem Analysis:}

The demand for Virtual Reality (VR) devices and applications increased heavily since the first consumer devices like \textit{HTC Vive} and \textit{Oculus Rift} were released during last years. But the development of the first Head-Mounted-Display dates back to 1966 and was developed by Sutherland and Goertz. 

One main difficulty of the current development of VR-applications is the lack of standardisation of the Software Development Kit (SDK) and interfaces. The most acknowledged suppliers \textit{HTC} and \textit{Oculus} do not work together or force standards for VR application development. Thus, all applications are system related and incompatible with other devices. Accordingly, each device offers different opportunities of interaction methods. These methods could be divided in the acknowledged categories selecting, grabbing, manipulating, movement and indirect controlling via widgets, gestures and voice input. Several suppliers currently offer different devices for interaction. And with focus on the grabbing and positioning methods, the most common are the \textit{Oculus}-HMD, \textit{HTC Vive}-HMD, \textit{HTC Vive}-Controller, data gloves and motion capturing systems for hand-tracking like the \textit{LeapMotion}-Controller.

Currently there exist no interaction laboratory that compares the different methods of interaction with objects, e.g. different methods for grabbing objects far away. Similar laboratories \cite{lin2016towards}\cite{website:TU}\cite{website:steam} exist where the user can experience virtual reality in different settings but the user can not compare different methods of interaction. This circumstance claims a virtual laboratory, where different interaction methods could be compared, demonstrated or tested in virtual test scenes. Thus, user friendly interaction methods which are not tiring and do not destroy the immersion could be improved by researcher. Another aspect of those methods is the increasing usability of VR applications and potential consumer will prefer devises with their implementation. Therefore, the profit of VR device suppliers will be squeezed. \\

\textbf{Objective and Requirements:}

A virtual interface laboratory is required for the development of an environment to test and compare interaction methods and to develop new ones. A further aspect of the laboratory is the use for teaching purposes or to give students a tool for the technical realisation of interaction studies in virtual (or augmented) reality environments. Thus, one task is to develop sophisticated test scenes for testing the interaction methods. These scenes should implement test exercises in different difficulty levels and represent typical and well-known environments like shops. All relevant parameters for the evalutation of the methods should be measured and saved in an output file.

\textbf{Solution Concept}
 
  The Agent provides a concept of an interaction laboratory for grabbing and positioning interactions at close or far range. It includes two test scenes, where the first is a practise room, in which the users can get familiar with the interaction methods. The second scene is designed as a supermarket. This environment was chosen because it offers various possibilities of exercises under changing difficulties like grabbing small mushrooms, fetching distantly placed tins or putting goods into boxes or shelves. The exercises are offered in form of a shopping list that tells the participant what goods have to be grabbed and repositioned. These various shopping list are predefined and all goods have to put into a shopping basket, for example. 
  
  Additional, the system offers a measurement of the accuracy as well as a time measuring of duration for every performed task. Furthermore, there will be a questionnaire designed to give the users an usability evaluation tool usability at hand. This questionnaire will test usability parameters as tiring, learnability, self-descriptiveness and fulfilling expectations.
 
 All rooms are implemented in Unity and the VR components are controlled by the same framework. Further, the \textit{HTC Vive}-HMD and the corresponding controllers are used to run the interactions, imaging and orientation in the environment. It is planned to realise at least six interaction methods of grabbing and positioning. Additional, the complete framework should be compatible with new test scenes and other interaction categories. \\

 
\textbf{Services}

The agent is bound to deliver a scientific project documentation in English language. Each project member is called upon to write her own chapter about her contributions. This documentation will be handed in in form of a PDF file and in printed form twice. 

A final project presentation is required in September or October. This presentation includes a 30 minutes talk, a live demonstration of the final result and a following discussion.

Furthermore the delivery of all resulting data files like the program code, content and artwork is required. This data has to be copied on a USB-Stick or in a cloud folder with appropriate access rights.

A brief project profile is inquired which includes a short description, two or more significant images and an optional video.\\


\textbf{Budget:}\\

\begin{tabular}{| l | l |}
\hline
\textbf{Material} & \textbf{Kosten in EUR}\\\hline
 Unity IDE  & 0\\ \hline
 Steam  & 0\\ \hline
 Visual Studio IDE  & 0\\ \hline
 MS Office  & 0 (TH Cologne license)\\ \hline
LateX  & 0 \\ \hline
 Github Repository & 0\\ \hline
 Projectplace - Projectmanagement Web Application& 0\\ \hline
 Google Drive Folder (Memory Space in Cloud) & 0\\ \hline
 3D Object Assets of furniture and goods & 0\\ \hline
  HTV Vive, 2 light houses and 2 Vive Controller & 900\\  \hline
 Computer for HTV Vive  & 1800 \\ \hline
 Photo and Video Camera  & private Camera \\ \hline
 

 \textbf{Gesamt} & 2700\\
 \hline
\end{tabular}
\bigskip

\textbf{Boundary Conditions:}\\

The first Prototype has to be presented on 01.06.2017 at 10:35 by the project manager. Furthermore, periodic meetings with the responsible professor are required and those have to be arranged via email during the lecture period of SS 2017. Unfortunately the facilities are only available at official business hours of the TH Cologne Building. \\

\textbf{Appointments and Milestones}\\

\begin{tabular}{|l | l |}
	\hline
	 Completion of the Project Planning & 24.04.2017\\\hline
	 First Prototype - Paper Mockup & 30.04.2017\\\hline
	 Second Prototype & 31.05.2017\\\hline
	 Third Prototype& 30.06.2017\\\hline
	 Completion of the documentation and video & 14.07.2017\\\hline
	 Final Release& 15.07.2017\\\hline
\end{tabular}

\bigskip
\bigskip
\bigskip
\bigskip
\bigskip
\bigskip


\begin{tabular}{ l l l }
.................................................................................. & & ..................................................................................\\
Date and Signature of Principal & &Date and Signature of Agent\\
\end{tabular}\\
\bibliographystyle{plain}
\renewcommand{\refname}{References}
\bibliography{main}
\end{document}